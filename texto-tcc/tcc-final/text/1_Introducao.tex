\chapter{Introdução}
\label{chp:introducao}

Em Dezembro de 2008, através da Lei nº 11.892, de 29 de Dezembro de 2008, foram criados os  Institutos Federais de Educação, Ciência e Tecnologia, que são instituições de educação superior, básica e profissional, pluricurriculares e \textit{multicampi}, especializados em oferecer educação profissional e tecnológica nas variadas formas de ensino, e equiparadas as universidades federais~\citep{lei11892}.

Os Institutos Federais estão espalhadas por todo o território brasileiro, havendo no mínimo mais de um instituto por estado, qualificando profissionais para os inúmeros setores da economia, realizando pesquisa e desenvolvendo novos processos, produtos e serviços~\citep{historiaif}. 

O \acf{IFG}, atualmente tem reitoria, sede e foro na cidade de Goiânia e possui 14 câmpus.
Entretanto, a história do IFG precede sua atual organização administrativa tendo nascido como Escolas de Aprendizes Artífices em 1909, que na época tinha o objetivo de capacitar os alunos em cursos e oficinas de forjas e serralheria, sapataria, alfaiataria, marcenaria e empalhação, selaria e correaria. 
Em 1942 virou Escola Técnica de Goiânia e teve a criação de cursos técnicos na área industrial, integrados ao ensino médio, por meio do Decreto-lei n.º 4.127, de 25 de fevereiro de 1942. 
Já em 1959 adquiriu a condição de autarquia federal e em 1965 tornou-se Escola Técnica federal de Goiás (ETFG)~\citep{historiaifg}. 
Em 1999 se tornou Centro Federal de Educação Tecnológica de Goiás (CEFET-GO), com autorização para ofertar cursos superiores, que no início tinha como política ser voltado para as "classes desprovidas", sendo uma instituição pública e gratuita. Para enfim em Dezembro de 2008, se transformar em Instituto Federal de Educação, Ciência e Tecnologia~\citep{historiaif}. 

Em sua organização administrativa atual, o IFG possui instâncias colegiadas, comissões, comitês e fóruns permanentes, reitoria, e os \textit{campi}. 
Incluída essas organizações administrativas, está a comunidade acadêmica, que é composta pelo corpo discente, docente e técnico-administrativo~\citep{regimentoifg}.
O corpo docente é composto pelos professores do quadro de pessoal definitivo do IFG, gerenciados pelo Regime Jurídico Único~\citep{brasil1990lei}, e demais professores admitidos na forma da legislação em vigor~\citep{regimentoifg}.

Nos \textit{campi}, os Departamentos de Áreas Acadêmicas (DAA) programam as atividades acadêmicas a serem desenvolvidas pelos servidores docentes, e distribuem a carga horária semanal de trabalho de acordo com o seu regime de trabalho, que podem ser de 20 horas semanais, 40 horas semanais, ou 40 horas semanais com dedicação exclusiva~\citep{resolucao}.

De acordo com Resolução 09 de 1º de Novembro de 2011, as atividades acadêmicas dos docentes são convertidas em pontos, e a somatória desses pontos devem totalizar a carga horária do regime de trabalho do servidor~\citep{resolucao}.
As atividades realizadas pelos docente compreendem ensino, pesquisa e extensão, e por isso diversos critérios foram criados nesta resolução para estabelecer a correlação entre as diversas atividades e a carga horária~\citep{resolucao}.

Obedecendo à Resolução 09 de 1º de Novembro de 2011~\citep{resolucao}, os docentes precisam apresentar um Plano Semestral de Trabalho e ao fim do semestre um Relatório Semestral de Trabalho.
O Plano Semestral de Trabalho tem sido preenchido através de um modelo em um documento no Microsoft Word$^{\circledR}$, e fazem suas somatórias de pontos em uma planilha Microsoft Excel$^{\circledR}$. 
Esse procedimento é propenso a problemas como diferentes formas de preenchimento de acordo com o entendimento do docente, não conformidade nas somas de pontos e falta de clareza em documento oficial.
Diante do exposto, este trabalho propõe uma solução para o docente preencher seu Plano de Trabalho Semestral. 
Tal solução tem duas principais características: uma interface amigável ao usuário e a garantia do respeito às normas da Resolução 09 de 1º de Novembro de 2011.

\section{Objetivo}
\indent

Construir uma solução \textit{user-friendly} e independente de sistema operacional para a elaboração do Plano Semestral de Trabalho nos termos da Resolução 09 de 1º de Novembro de 2011 do IFG.

\subsection*{Objetivos Específicos}

\begin{enumerate}
	\item Fazer a prototipação e documentação do sistema a partir do levantamento de requisitos.
	\item Aprovar o protótipo e desenvolver o sistema como uma aplicação web em conformidade com a resolução 09 de 1º de Novembro de 2011~\citep{resolucao}.
	\item Disponibilizar o sistema de forma pública para uso do IFG.
\end{enumerate}




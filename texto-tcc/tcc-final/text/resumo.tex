O \acf{IFG} é uma instituição pública que oferta cursos de educação básica, profissional e superior. 
Os docentes do \ac{IFG} precisam apresentar semestralmente o ``Plano Semestral de Trabalho Docente'' com o planejamento das atividades a serem realizadas e suas respectivas cargas horárias e pontuações de acordo com a Resolução 09 de 1º de Novembro de 2011.
Todos os dados são convertidas em pontos de acordo com fatores de ponderação, sendo um instrumento de planejamento tanto do docente quanto do \acf{DAA} do \ac{IFG}. 
Entretanto, os planos de trabalho semestrais são baseados em modelos de documentos de texto e planilha eletrônica.
Esse cenário é propício para problemas de preenchimento que levam frequentemente a planos de trabalho em desacordo com a norma.
Utilizando tecnologias atuais de mercado foi desenvolvida uma aplicação \textit{web} chamada ``\textit{e}-Plano''.
O \textit{e}-Plano foi desenvolvido para funcionar de forma independente de sistema operacional, responsivo ao tamanho da tela no \textit{browser} e com interface amigável ao usuário.
Tais características foram fruto da aplicação de diversas técnicas e metodologias de desenvolvimento de software, com vistas a facilitar o preenchimento do ``Plano Semestral de Trabalho Docente'' garantindo sua conformidade com a Resolução 09 de 1º de Novembro de 2011.

\begin{keywords}
Plano Semestral de Trabalho Docente, IFG, desenvolvimento de software, aplicação \textit{Web} 
\end{keywords}